%Dokumentacja techniczna projektu Rękawica Sensoryczna 2017
\usepackage{graphicx}
\usepackage{hyperref}



\renewcommand{\maketitle}{\begin{titlepage}

\begin{center}
%%\includegraphics[scale=1]{logo.png}
%\vspace*{1cm}
%\noindent \rule{\linewidth}{0.4mm}
%\vspace*{0,5cm}

%\LARGE \textsc{Raport}\\
%\large \textsc{Bazy Danych - Projekt}\\
%\small\textsc{pon. 9:15--11:00}\\
%\small\textsc{dr hab. inż. Grzegorz Mzyk}\\

%\vspace*{0.5cm}
%\Large \textsc{Edytor rozdań brydżowych}

%\vspace*{0.5cm}
%\rule{\linewidth}{0.4mm}
%\vspace*{2cm}

%\Large\textsc{Anna Postawka} \\ % W tym miejscu należy zamieścić 
%\large\textsc{218556}\\ % listę członków zespołu projektowego
%\Large\textsc{Aleksander Sil} \\ % W tym miejscu należy zamieścić
%\large\textsc{218576}\\ % listę członków zespołu projektowego

\vspace*{5cm}
\LARGE\textsc{Rękawica Sensoryczna}\\
\LARGE\textsc{Dokumentacja techniczna}\\

%\large\centering Projekt realizowany w ramach kursu Roboty Mobilne 1 na Politechnice Wrocławskiej\\
\vspace{3cm}
\normalsize\flushleft\textbf{Temat Projektu:} Rękawica sensoryczna\\
\textbf{Autorzy:} Krzysztof Dąbek 218549, Dymitr Choroszczak 218627,\\Anna Postawka 218556\\
\textbf{Kierunek:} Automatyka i Robotyka\\
\textbf{Specjalność:} Robotyka (ARR)\\
\textbf{Prowadzący:} dr inż. Andrzej Wołczowski\\
\textbf{Kurs:} Roboty Mobilne 1\\
\textbf{Termin zajęć:} pn TN 11:15, śr TN 14:30\\
\vspace{5 mm}


\vspace*{7cm}
\centering\textsc{\today}\\

\end{center}
\end{titlepage}
\newpage
}