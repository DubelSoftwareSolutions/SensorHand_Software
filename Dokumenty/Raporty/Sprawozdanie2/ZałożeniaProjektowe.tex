\documentclass[12pt,a4paper]{article}
\usepackage[utf8x]{inputenc}
\usepackage{ucs}
\usepackage[MeX]{polski}
\usepackage{fancyhdr}
\usepackage{amsmath}
\usepackage{amsfonts}
\usepackage{amssymb}
\pagestyle{fancy}
\usepackage{enumerate}
\usepackage{listings}
\usepackage{graphicx}
\usepackage{subfig}
\usepackage{caption}
\usepackage{float}
\usepackage{tabularx}
\begin{document}
\LARGE\centering Sprawozdanie z postępów prac nad projektem Rękawicy Sensorycznej\\
\large\centering Projekt realizowany w ramach kursu Roboty Mobilne 1 na Politechnice Wrocławskiej\\
\vspace{5 mm}
\normalsize\flushleft\textbf{Temat Projektu:} Rękawica sensoryczna\\
\textbf{Autorzy:} Krzysztof Dąbek 218549, Dymitr Choroszczak 218627,\\Anna Postawka 218556\\
\textbf{Kierunek:} Automatyka i Robotyka\\
\textbf{Specjalność:} Robotyka (ARR)\\
\textbf{Prowadzący:} dr inż. Andrzej Wołczowski\\
\textbf{Kurs:} Roboty Mobilne 1\\
\textbf{Termin zajęć:} pn TN 11:15, śr TN 14:30\\
\vspace{5 mm}
\section{Ukończone zadania}
\subsection{Odczyty z czujników}

\subsection{Rękawica sensoryczna}
Został wykonany prototyp rękawicy sensorycznej. 
Tak jak na schemacie [?], 5 czujników mierzących ugięcie palców zostało przyszytych na wierzchu dłoni, a jeden czujnik mierzący skręt kciuka został przyszyty wewnątrz dłoni. Czujniki nacisku zostały przyklejone na opuszkach. 
Czujniki zostały przyszyte do rękawicy.
Robocza rękawica z materiału typu ... jednak się nie sprawdza, trzeba będzie prawdopodobnie wykonać nową.
Prototyp nadaje się do początkowych testów, ale żeby móc przeprowadzać dłuższe, intensywniejsze pomiary, trzeba będzie wykonać nowy prototyp.
\section{Wizualizacja}

\section{Krytyczna ocena działań}


\end{document}
