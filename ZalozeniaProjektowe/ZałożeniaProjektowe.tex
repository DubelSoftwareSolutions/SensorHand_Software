\documentclass[12pt,a4paper]{article}
\usepackage[utf8x]{inputenc}
\usepackage{ucs}
\usepackage[MeX]{polski}
\usepackage{fancyhdr}
\usepackage{amsmath}
\usepackage{amsfonts}
\usepackage{amssymb}
\pagestyle{fancy}
\usepackage{enumerate}
\usepackage{listings}
\usepackage{subfig}
\begin{document}
\LARGE\centering Założenia projektu Roboty Mobilne\\
\large\centering Projekt realizowany w ramach kursu Roboty Mobilne 1 na Politechnice Wrocławskiej\\
\vspace{5 mm}
\normalsize\flushleft\textbf{Temat Projektu:} Rękawica sensoryczna\\
\textbf{Autorzy:} Krzysztof Dąbek 218549, Dymitr Choroszczak 218627,\\Anna Postawka 218556\\
\textbf{Kierunek:} Automatyka i Robotyka\\
\textbf{Specjalność:} Robotyka (ARR)\\
\textbf{Prowadzący:} dr inż. Andrzej Wołczowski\\
\textbf{Kurs:} Roboty Mobilne 1\\
\textbf{Termin zajęć:} pn TN 11:15, śr TN 14:30\\
\vspace{5 mm}
\section{Główne założenia projektowe: }\normalsize
\begin{itemize}
\item Stworzenie rękawicy z czujnikami ugięcia w trzech palcach oraz czujnikami nacisku na opuszkach
\item Czujnik zgięcia dłoni na wewnętrznej stronie rękawicy
\item Zamontowanie na opuszkach LEDów (np. RGB) wizualizujących odczyty z czujników nacisku
\item Wykorzystanie płytki STM32F3Discovery do przetwarzania danych
\item Użycie akcelerometru zawartego na płytce do określenia położenia dłoni względem pionu (przyśpieszenia grawitacyjnego)
\item Bezprzewodowe przesyłanie danych do komputera za pomocą modułu Bluetooth HC-06
\item Przewodowe przesyłanie danych do komputera za pomocą interfejsu USB
\item Zewnętrzne zasilanie z akumulatora
\end{itemize}
Projekt zostanie połączony z innym realizowanym w ramach kursu Wizualizacja Danych Sensorycznych. Dane z sensorów rękawicy posłużą do stworzenia uproszczonego modelu dłoni w wizualizacji 3D.
\subsection{Opis czujników}
\begin{itemize}
\item Na opuszkach palców zamontowane zostaną czujniki siły nacisku FSR-400. Spadek rezystancji przy przyłożonej sile pozwala zmierzyć siłę nacisku.
\item Do wykrycia zgięcia stawów międzypaliczkowych bliższych oraz stawu międzypaliczkowego kciuka zastosowane zostaną czujniki ugięcia - flexsensory firmy Sparkfun. Zgięcie tych sensorów powoduje wzrost rezystancji.
\item Gumowy przewód przewodzący prąd do pomiaru siły rozciągania firmy Adafruit wykorzystany zostanie do wykrycia zgięcia stawów śródręczno-paliczkowych dłoni. Rezystancja tego elementu wzrasta wraz z jego rozciąganiem.
\item Akcelerometr LSM303DLHC, znajdujący się na płytce Discovery zostanie użyty do określenia orientacji rękawicy względem wektora grawitacji.
\end{itemize}
\section{Harmonogram pracy}
\begin{enumerate}
\item (22.03.2017) Zakup elementów potrzebnych do konstrukcji 
\item (30.03.2017) Nawiązanie łączności płytki z komputerem (USB/Bluetooth) 
\item (13.04.2017) Zaprogramowanie odczytu danych z czujników (tensometrów, czujników nacisku oraz akcelerometru) 
\item (27.04.2017) Montaż czujników oraz płytki na rękawicy 
\item (11.05.2017) Oprogramowanie wstępnego przetwarzania danych przez płytkę 
\item (25.05.2017) Przygotowanie gotowej aplikacji 
\item (01.06.2017) Przeprowadzenie testów oraz wprowadzenie ewentualnych poprawek 
\item (15.06.2017) Dokumentacja projektu 
\end{enumerate}
\end{document}
